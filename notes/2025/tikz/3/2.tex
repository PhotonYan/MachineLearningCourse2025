\documentclass[tikz,border=6pt]{standalone}
\usepackage{amsmath}
\usetikzlibrary{arrows.meta}

% ===== 可调参数 =====
\def\slope{0.6}     % 直线斜率:y = slope*x + intercept
\def\intercept{0.0} % 截距 b 对应 y 截距(控制超平面位置)
\def\gap{1.0}       % “功能间隔”对应的代数位移(±1),几何宽度=2/||w||
\def\xmin{-4.5}\def\xmax{4.5}
\def\ymin{-3.5}\def\ymax{3.5}

% ||w||,若写成 y = slope*x + intercept,则可取 w = (slope, -1)
\pgfmathsetmacro{\wnorm}{sqrt(\slope*\slope + 1.0)}
\pgfmathsetmacro{\halfwidth}{1.0/\wnorm} % 单侧几何宽度

\begin{document}
\begin{tikzpicture}[>=Stealth, line cap=round, line join=round,scale=0.8]
  % 轴
  
  \draw[->,thin] (\xmin,0) -- (\xmax,0) node[below left] {$x_1$};
  \draw[->,thin] (0,\ymin) -- (0,\ymax) node[below left] {$x_2$};
  \node at (5.9,1.8) {$w^\top x+b=-1$};
  \node at (5.7,2.7) {$w^\top x+b=0$};
  \node at (5.4,3.6) {$w^\top x+b=1$};
  \begin{scope}
    \clip (-3.8,-3.3)--(4.2,1.5)--(4.2,-3.3)--cycle;
    \fill[color=red,opacity=0.2] (-4.3,-3.3)rectangle(4.3,3.3);
  \end{scope}
  \begin{scope}
    \clip (3.8,3.3)--(-4.2,-1.5)--(-4.2,3.3)--cycle;
    \fill[color=blue,opacity=0.2] (-4.3,-3.3)rectangle(4.3,3.3);
  \end{scope}
  
  {
  \clip (-4.3,-3.3)rectangle(4.3,3.3);


  % 超平面与两条边界线
  \draw[very thick] (\xmin,\slope*\xmin+\intercept) -- (\xmax,\slope*\xmax+\intercept);

  \draw[dashed, thick] (\xmin,\slope*\xmin+\intercept+\gap) -- (\xmax,\slope*\xmax+\intercept+\gap)
    node[anchor=west] {$w^\top x+b=+1$};
  \draw[dashed, thick] (\xmin,\slope*\xmin+\intercept-\gap) -- (\xmax,\slope*\xmax+\intercept-\gap);

  % 法向量 w(示意)
  \draw[->,thick] (0,0) -- ({1.2*\slope},{-1.2}) node[below right] {$w$};

  % 几何宽度标注(在中线选点 P,沿法向量正交画出 2/||w||)
  \pgfmathsetmacro{\Px}{2.0}
  \pgfmathsetmacro{\Py}{\slope*\Px+\intercept}
  \pgfmathsetmacro{\ux}{\slope/\wnorm}  % 单位法向量分量
  \pgfmathsetmacro{\uy}{-1.0/\wnorm}
  \coordinate (A) at ({\Px - \ux*\halfwidth},{\Py - \uy*\halfwidth});
  \coordinate (B) at ({\Px + \ux*\halfwidth},{\Py + \uy*\halfwidth});
  \draw[<->] (A) -- (B) node[midway, xshift=3pt,fill=white,opacity=0.8] {$\frac{2}{\lVert w\rVert}$};

  % ===== 样本点 =====
  % +1 类(蓝色,满足 w^T x + b >= 0,粗略放置在线下方)
  \begin{scope}[shift={(0,1.7)}]
    \foreach \xx/\yy in {-3/-2.1, -2.3/-1.6, -1.6/1.0, -1/-0.8, -0.4/0.5,
                       0.6/0.0, 1.1/0.2, 1.8/1.55, 2.6/1.0, 3.2/1.4} {
    \fill[blue!70] (\xx,\yy) circle (2pt);
  }
  \end{scope}
  
  % 贴在上边界的支持向量(圈出来)
  \foreach \xx in {-1.8, 0.7, 2.1} {
    \pgfmathsetmacro{\yy}{\slope*\xx+\intercept+\gap}
    \fill[blue!70] (\xx,\yy) circle (2pt);
    \draw[blue!70, very thick] (\xx,\yy) circle (4pt);
  }

  % -1 类(红色,满足 w^T x + b <= 0,粗略放置在线上方)
  \begin{scope}[shift={(0,-3.5)}]
    \foreach \xx/\yy in {-3/0.4, -2.4/0.9, -1.5/1.2, -0.6/1.1, 0.2/1.2,
                       1.0/1.6, 1.9/2.6, 2.8/2.3, 3.4/3.5} {
    \fill[red!70] (\xx,\yy) circle (2pt);
  }
  \end{scope}
  
  % 贴在下边界的支持向量(圈出来)
  \foreach \xx in {-2.2, -0.5, 1.6} {
    \pgfmathsetmacro{\yy}{\slope*\xx+\intercept-\gap}
    \fill[red!70] (\xx,\yy) circle (2pt);
    \draw[red!70, very thick] (\xx,\yy) circle (4pt);
  }

  % 区域标注(可删)
  \node[red!70!black] at (2.8,-2.2) {\small $y=+1$};
  \node[blue!70!black]  at (-2.8,2) {\small $y=-1$};

  % 说明(可删)
%   \node[align=left, anchor=west] at (\xmin+0.1,\ymin+0.5) {\footnotesize
%     \(\begin{aligned}
%       &\text{Hyperplane: } w^\top x+b=0\\
%       &\text{Margins: } w^\top x+b=\pm1\\
%       &\text{Width: } \dfrac{2}{\|w\|}
%     \end{aligned}\)
%   };

  % 裁剪边界
  \path[use as bounding box] (\xmin,\ymin) rectangle (\xmax,\ymax);
  }
\end{tikzpicture}
\end{document}